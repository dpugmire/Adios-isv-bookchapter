\newpage
\section{Outline}
*** There is no page limit but should be about 15-20 pages in the Springer format, which is about 8 pages in double column format***
Due date: 27 Jan 2020

Here is a working outline for the chapter.


\begin{itemize}
\item Motivate ADIOS and point out the benefits it provides under different circumstances:
\begin{itemize}
\item File-like abstraction for data movement
\item Engines that use the same API, but behave differently.
\item Description (table?) of each engine and their characteristics/benefits
\begin{itemize}
\item Possible characteristics (from the Kressian ISAV paper)
\begin{itemize}
\item data: access, movement, duplication
\item coordination
\item resource requirements
\item exploratory vis
\item resilience / fault tolerance
\item ease of use
\end{itemize}
\end{itemize}
\item Compression
\item ADIS
\end{itemize}
\item Describe several uses cases. Go from simple to more complex.
\begin{itemize}
\item Inline visualization. Simple case: PDF on particles
\item Example with light comm (contour? James' stuff?)
\item XGC and FTK (requires comm, SST)
\item Strong coupling XGC-GENE, SSC, SST or BP4 (discuss +/- of each engine choice) Eric, Jong
\item Weak coupling: ??, ?? (discuss +/- of each engine choice) Eric, Jong
\item KSTAR: remote pure push. Most complex example. DataMan.
\item SpecFEM3D? --probably not
\item JAXA --Most complex example
\item Placement (James' stuff)
\end{itemize}
\item Conclusion that restates characteristics/benefits of the many different configurations ADIOS has available. 
\end{itemize}

